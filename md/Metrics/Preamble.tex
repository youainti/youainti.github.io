
%%%%%%%%%Packages%%%%%%%%%%%%%%%
\usepackage{biblatex}
\usepackage{amsmath}
\usepackage{mathtools}
\usepackage{amsthm}
\usepackage{amssymb}
\usepackage{thmtools, thm-restate}
\usepackage{cleveref}




%%%%%%%%%%%%SETUP THEOREMS%%%%%%%%%%%%%%%%%%%%%%
\declaretheorem[within=subsection]{theorem}





%%%%%%%%%%%%FORMATTING%%%%%%%%%%%%%%%%%%%%%
%Math formatting
\newcommand{\bb}[1]{\mathbb{#1}}
\newcommand{\induction}[3]{
        \textbf{#1}\\
        \underline{Basis Step:}\\
        #2

        \underline{Induction Step:}\\
        #3
}
\newcommand{\mle}[1]{\hat{#1}_{\text{MLE}}}
\newcommand{\pr}[1]{\text{Pr}\left( #1 \right)}
\newcommand{\thm}[3]{
\begin{restatable}[#1]{theorem}{#2}
\label{thm:#2}
#3
\end{restatable}
}

%These are some formatting/reminder commands
\newcommand{\todo}[1]{
\textbf{\#TODO: \underline{#1}}
}
\newcommand{\define}[1]{%for now just highlight, in future, link.
	\underline{\textbf{#1}}
}
\newcommand{\mittle}[1]{\footnote{Mittlehammer pg. #1}}
\newcommand{\likeli}[2]{\text{L}\left( #1 | #2 \right)}


%%%%%%%%%%%%%%%Math Operators%%%%%%%%%%%%%%%%%%
\DeclareMathOperator{\argmax}{argmax}
\DeclareMathOperator*{\plim}{plim}

%%%%%%%%%%%%%%INTERNAL REFERENCES%%%%%%%%%%%%%%
%\newcommand{\eqref}[1]{Eq. \ref{#1}} %Already defined


%%%%%%%%%%%%CITATIONS%%%%%%%%%%%%%%%%%
%%%%% Adjust this at some point.
%\bibliography{/home/will2/Desktop/Repos/homework_repos/Math461/References.bib}
%This is how to perform citations.
% Use \cite{ref} to get a numerical reference to the bibliography
% Use \citetitle{ref} to get the title as a citation (Use for most cases)
% Use \fullcite{ref} to insert the full reference.
% NOTE: I am running pdflatex > biber > pdflates > pdflatex
\newcommand{\numcite}[1]{\textsuperscript{\cite{#1}}}
\newcommand{\nameref}[1]{\citetitle{#1}\numcite{#1}}


\author{Will King}



